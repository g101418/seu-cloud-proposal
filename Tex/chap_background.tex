\section{研究背景}

森林和草原是重要的自然生态系统,对于维护国家生态安全、推进生态文明建设具有基础性、战略性作用。为全面提升森林和草原等生态系统功能,进一步压实地方各级党委和政府保护发展森林草原资源的主体责任,现就全面推行林长制提出以下意见。

\insertgraph{tableexample.pdf}{表格示例}{Fig:tableexample}[0.4]

各地要综合考虑区域、资源特点和自然生态系统完整性\cite{Chu2014Discovering}\cite{2005What},各级林长组织领导责任区域森林草原资源保护发展工作\cite{1996How}\cite{2005What},落实保护发展森林草原资源目标责任制\cite{ziemann2016gene}。将森林覆盖率、森林蓄积量、草原综合植被盖度、沙化土地治理面积等作为重要指标\cite{Tibbetts}。因地制宜确定目标任务;组织制定森林草原资源保护发展规划计划,强化统筹治理,推动制度建设,完善责任机制组织协调解决责任区域的重点难点问题。

以习近平新时代中国特色社会主义思想为指导:全面贯彻党的十九大和十九届二中\cite{Bohannon2005A}三中\cite{Chu2013Holistic}四中\cite{hao2017cleaning}五中全会精神\cite{Hao2018Distilling}、认真践行习近平生态文明思想\cite{Rekatsinas2017HoloClean}和坚定贯彻新发展理念\cite{Yakout2011Guided}。指导思想。以习近平新时代中国特色社会主义思想为指导,全面贯彻党的十九大和十九届二中、三中、四中、五中全会精神,认真践行习近平生态文明思想,坚定贯彻新发展理念,根据党中央、国务院决策部署,按照山水林田湖草系统治理要求,在全国全面推行林长制,明确地方党政领导干部保护发展森林草原资源目标责任,构建党政同责、属地负责、部门协同、源头治理、全域覆盖的长效机制,加快推进生态文明和美丽中国建设。

加强森林草原资源生态保护。严格森林草原资源保护管理,严守生态保护红线。严格控制林地、草地转为建设用地,加强重点生态功能区和生态环境敏感脆弱区域的森林草原资源保护

\begin{enumerate}
    \item \textbf{关于全面推行林长制的意见:}森林和草原是重要的自然生态系统,对于维护国家生态安全、推进生态文明建设具有基础性、战略性作用。为全面提升森林和草原等生态系统功能,进一步压实地方各级党委和政府保护发展森林草原资源的主体责任,现就全面推行林长制提出以下意见。
    \item \textbf{总体要求:}指导思想。以习近平新时代中国特色社会主义思想为指导,全面贯彻党的十九大和十九届二中、三中、四中、五中全会精神,认真践行习近平生态文明思想,坚定贯彻新发展理念,根据党中央。
    \item \textbf{工作原则:}国务院决策部署,按照山水林田湖草系统治理要求,在全国全面推行林长制,明确地方党政领导干部保护发展森林草原资源目标责任,构建党政同责、属地负责、部门协同、源头治理、全域覆盖的长效机制,加快推进生态文明和美丽中国建设。
\end{enumerate}

坚持生态优先、保护为主。全面落实森林法、草原法等法律法规,建立健全最严格的森林草原资源保护制度,加强生态保护修复,保护生物多样性,增强森林和草原等生态系统稳定性。

\challenge{挑战一}{坚持绿色发展、生态惠民}

坚持党委领导、部门联动。加强党委领导,建立健全以党政领导负责制为核心的责任体系,明确各级林(草)长(以下统称林长)的森林草原资源保护发展职责,强化工作措施,统筹各方力量,形成一级抓一级、层层抓落实的工作格局。

\textbf{坚持问题导向、因地制宜}。组织体系。各省(自治区、直辖市)设立总林长,由省级党委或政府主要负责同志担任;设立副总林长,由省级负责同志担任,实行分区(片)负责。各省(自治区、直辖市)根据实际情况,可设立市、县、乡等各级林长。地方各级林业和草原主管部门承担林长制组织实施的具体工作。

综上所述,工作职责。各地要综合考虑区域、资源特点和自然生态系统完整性,科学确定林长责任区域。各级林长组织领导责任区域森林草原资源保护发展工作,落实保护发展森林草原资源目标责任制,将森林覆盖率、森林蓄积量、草原综合植被盖度、沙化土地治理面积等作为重要指标,因地制宜确定目标任务;组织制定森林草原资源保护发展规划计划。

\challenge{挑战二}{加强森林草原资源生态保护}

严格森林草原资源保护管理,严守生态保护红线。严格控制林地、草地转为建设用地,加强重点生态功能区和生态环境敏感脆弱区域的森林草原资源保护,禁止毁林毁草开垦。加强公益林管护,统筹推进天然林保护,全面停止天然林商业性采伐,完善森林生态效益补偿制度。落实草原禁牧休牧和草畜平衡制度,完善草原生态保护补奖政策。强化森林草原督查,严厉打击破坏森林草原资源违法犯罪行为。推进构建以国家公园为主体的自然保护地体系。强化野生动植物及其栖息地保护。

加强森林草原资源生态修复。依据国土空间规划,科学划定生态用地,持续推进大规模国土绿化行动。实施重要生态系统保护和修复重大工程,推进京津冀协同发展、长江经济带发展、粤港澳大湾区建设、长三角一体化发展、黄河流域生态保护和高质量发展、海南自由贸易港建设等重大战略涉及区域生态系统保护和修复,深入实施退耕还林还草、三北防护林体系建设、草原生态修复等重点工程。加强森林经营和退化林修复,提升森林质量。落实部门绿化责任,创新义务植树机制,提高全民义务植树尽责率。

综上所述,加强森林草原资源灾害防控。建立健全重大森林草原有害生物灾害防治地方政府负责制,将森林草原有害生物灾害纳入防灾减灾救灾体系。

针对上述挑战深化森林草原领域改革。巩固扩大重点国有林区和国有林场改革成果,加强森林资源资产管理,推动林区林场可持续发展:

\begin{enumerate}
    \item \textbf{加强森林草原资源监测监管:}充分利用现代信息技术手段,不断完善森林草原资源“一张图”、“一套数”动态监测体系,逐步建立重点区域实时监控网络,及时掌握资源动态变化,提高预警预报和查处问题的能力,提升森林草原资源保护发展智慧化管理水平。
    \item \textbf{加强基层基础建设:}充分发挥生态护林员等管护人员作用,实现网格化管理。加强乡镇林业(草原)工作站能力建设,强化对生态护林员等管护人员的培训和日常管理。建立市场化、多元化资金投入机制,完善森林草原资源生态保护修复财政扶持政策。
\end{enumerate}