\section{研究内容}

为了实现上述的研究目标,加强基层基础建设。充分发挥生态护林员等管护人员作用,实现网格化管理。加强乡镇林业(草原)工作站能力建设,强化对生态护林员等管护人员的培训和日常管理。建立市场化、多元化资金投入机制,完善森林草原资源生态保护修复财政扶持政策,如图\ref{Fig:content}所示:

首先加强组织领导。地方各级党委和政府是推行林长制的责任主体,要切实强化组织领导和统筹谋划,明确责任分工,细化工作安排,狠抓责任落实,确保到2022年6月全面建立林长制。

其次健全工作机制。建立健全林长会议制度、信息公开制度、部门协作制度、工作督查制度,研究森林草原资源保护发展中的重大问题,定期通报森林草原资源保护发展重点工作。

最后接受社会监督。建立林长制信息发布平台,通过媒体向社会公告林长名单,在责任区域显著位置设置林长公示牌。有条件的地方可以推行林长制实施情况第三方评估。每年公布森林草原资源保护发展情况。加强生态文明宣传教育,增强社会公众生态保护意识,自觉爱绿植绿护绿。