\section{技术路线}

加强森林草原资源生态保护。严格森林草原资源保护管理,严守生态保护红线。严格控制林地、草地转为建设用地,加强重点生态功能区和生态环境敏感脆弱区域的森林草原资源保护,禁止毁林毁草开垦。加强公益林管护,统筹推进天然林保护,全面停止天然林商业性采伐,完善森林生态效益补偿制度。落实草原禁牧休牧和草畜平衡制度,完善草原生态保护补奖政策。强化森林草原督查,严厉打击破坏森林草原资源违法犯罪行为。推进构建以国家公园为主体的自然保护地体系。强化野生动植物及其栖息地保护。

\subsection{加强森林草原资源灾害防控}

建立健全重大森林草原有害生物灾害防治地方政府负责制,将森林草原有害生物灾害纳入防灾减灾救灾体系,健全重大森林草原有害生物监管和联防联治机制,抓好松材线虫病、美国白蛾、草原鼠兔害等防治工作。坚持森林草原防灭火一体化,落实地方行政首长负责制,提升火灾综合防控能力。

深化森林草原领域改革。巩固扩大重点国有林区和国有林场改革成果,加强森林资源资产管理,推动林区林场可持续发展。完善草原承包经营制度,规范草原流转。深化集体林权制度改革,鼓励各地在所有权、承包权、经营权“三权分置”和完善产权权能方面积极探索,大力发展绿色富民产业,所示:推动林区林场可持续发展。完善草原承包经营制度,规范草原流转。深化集体林权制度改革,鼓励各地在所有权、承包权、经营权“三权分置”和完善产权权能方面积极探索,大力发展绿色富民产业。


所以,加强森林草原资源监测监管。充分利用现代信息技术手段的技术路线如下:

\subtitle{1)}{加强森林草原资源监测监管}

不断完善森林草原资源“一张图”、“一套数”动态监测体系,逐步建立重点区域实时监控网络,及时掌握资源动态变化,提高预警预报和查处问题的能力,提升森林草原资源保护发展智慧化管理水平。

加强基层基础建设。充分发挥生态护林员等管护人员作用,实现网格化管理。加强乡镇林业(草原)工作站能力建设,强化对生态护林员等管护人员的培训和日常管理。建立市场化、多元化资金投入机制,完善森林草原资源生态保护修复财政扶持政策。

\subtitle{2)}{加强组织领导}

地方各级党委和政府是推行林长制的责任主体,要切实强化组织领导和统筹谋划,明确责任分工,细化工作安排,狠抓责任落实,确保到2022年6月全面建立林长制。

工作职责。各地要综合考虑区域、资源特点和自然生态系统完整性,科学确定林长责任区域。各级林长组织领导责任区域森林草原资源保护发展工作,落实保护发展森林草原资源目标责任制,将森林覆盖率、森林蓄积量、草原综合植被盖度、沙化土地治理面积等作为重要指标,因地制宜确定目标任务;组织制定森林草原资源保护发展规划计划,强化统筹治理,推动制度建设,完善责任机制;组织协调解决责任区域的重点难点问题,依法全面保护森林草原资源,推动生态保护修复,组织落实森林草原防灭火、重大有害生物防治责任和措施,强化森林草原行业行政执法。

组织体系。各省(自治区、直辖市)设立总林长,由省级党委或政府主要负责同志担任;设立副总林长,由省级负责同志担任,实行分区(片)负责。各省(自治区、直辖市)根据实际情况,可设立市、县、乡等各级林长。地方各级林业和草原主管部门承担林长制组织实施的具体工作。

\subsection{面向表格融合知识图谱的错误修复}

通过上节模型对表格数据进行初步异常发现后,从规模庞大的表格中得以筛选数量较少的疑似错误集,一定程度上缓解了运行时效率的压力。在本节,将介绍表格数据中融合知识图谱的错误修复办法。该部分面临两个困难点:\textbf{1)表格格式处理难。}表格格式数据有不同于常见文本描述的形式,其语义被行和列的规划所约束和表示,如何针对表格中行、列的特点,针对性的设计模型以处理表格数据,是一个难点。\textbf{2)知识图谱融合难。}知识图谱富含丰富的背景知识,将图谱的只是引入模型,可以大大提高知识推断或者说错误修复的准确性,结果也更具有解释性,通过特殊的模型结构,将图谱融入修复过程,是一大难题。

针对上述两个难点,本硕士论文的解决思路分为:

\subtitle{1)}{坚持生态优先、保护为主。全面落实森林法}

指导思想。以习近平新时代中国特色社会主义思想为指导,全面贯彻党的十九大和十九届二中、三中、四中、五中全会精神,认真践行习近平生态文明思想,坚定贯彻新发展理念,根据党中央、国务院决策部署,按照山水林田湖草系统治理要求,在全国全面推行林长制,明确地方党政领导干部保护发展森林草原资源目标责任,构建党政同责、属地负责、部门协同、源头治理、全域覆盖的长效机制,加快推进生态文明和美丽中国建设。

健全工作机制。建立健全林长会议制度、信息公开制度、部门协作制度、工作督查制度,研究森林草原资源保护发展中的重大问题,定期通报森林草原资源保护发展重点工作。

接受社会监督。建立林长制信息发布平台,通过媒体向社会公告林长名单,在责任区域显著位置设置林长公示牌。有条件的地方可以推行林长制实施情况第三方评估。每年公布森林草原资源保护发展情况。加强生态文明宣传教育,增强社会公众生态保护意识,自觉爱绿植绿护绿。

强化督导考核。林长制督导考核纳入林业和草原综合督查检查考核范围,县级及以上林长负责组织对下一级林长的考核,考核结果作为地方有关党政领导干部综合考核评价和自然资源资产离任审计的重要依据。落实党政领导干部生态环境损害责任终身追究制,对造成森林草原资源严重破坏的,严格按照有关规定追究责任。

加强组织领导。地方各级党委和政府是推行林长制的责任主体,要切实强化组织领导和统筹谋划,明确责任分工,细化工作安排,狠抓责任落实,确保到2022年6月全面建立林长制。

\subtitle{2)}{接受社会监督。建立林长制信息发布平台}

考核结果作为地方有关党政领导干部综合考核评价和自然资源资产离任审计的重要依据。

\indent\subtitle{(1)}{健全工作机制}

加强森林草原资源监测监管。充分利用现代信息技术手段,不断完善森林草原资源“一张图”、“一套数”动态监测体系,逐步建立重点区域实时监控网络,及时掌握资源动态变化,提高预警预报和查处问题的能力,提升森林草原资源保护发展智慧化管理水平。

加强基层基础建设。充分发挥生态护林员等管护人员作用,实现网格化管理。加强乡镇林业(草原)工作站能力建设,强化对生态护林员等管护人员的培训和日常管理。建立市场化、多元化资金投入机制,完善森林草原资源生态保护修复财政扶持政策。

\indent\subtitle{(2)}{加强组织领导}

深化森林草原领域改革。巩固扩大重点国有林区和国有林场改革成果,加强森林资源资产管理,推动林区林场可持续发展。完善草原承包经营制度,规范草原流转。深化集体林权制度改革,鼓励各地在所有权、承包权、经营权“三权分置”和完善产权权能方面积极探索,大力发展绿色富民产业。

加强森林草原资源灾害防控。建立健全重大森林草原有害生物灾害防治地方政府负责制,将森林草原有害生物灾害纳入防灾减灾救灾体系,健全重大森林草原有害生物监管和联防联治机制,抓好松材线虫病、美国白蛾、草原鼠兔害等防治工作。坚持森林草原防灭火一体化,落实地方行政首长负责制,提升火灾综合防控能力。