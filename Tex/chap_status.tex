\section{研究现状}

加强森林草原资源生态保护。严格森林草原资源保护管理,严守生态保护红线。严格控制林地、草地转为建设用地,加强重点生态功能区和生态环境敏感脆弱区域的森林草原资源保护,禁止毁林

\subsection{主要任务}

坚持党委领导、部门联动。加强党委领导,建立健全以党政领导负责制为核心的责任体系,明确各级林(草)长(以下统称林长)的森林草原资源保护发展职责,强化工作措施,统筹各方力量,形成一级抓一级、层层抓落实的工作格局\cite{chen2020table}。

组织体系。各省(自治区、直辖市)设立总林长,由省级党委或政府主要负责同志担任;设立副总林长,由省级负责同志担任,实行分区(片)负责。各省(自治区、直辖市)根据实际情况,可设立市、县、乡等各级林长。地方各级林业和草原主管部门承担林长制组织实施的具体工作。

坚持问题导向、因地制宜。针对不同区域森林和草原等生态系统保护管理的突出问题,坚持分类施策、科学管理、综合治理,宜林则林、宜草则草、宜荒则荒,全面提升森林草原资源的生态、经济、社会功能。

\subsection{工作职责}

指导思想。以习近平新时代中国特色社会主义思想为指导,全面贯彻党的十九大和十九届二中、三中、四中、五中全会精神,认真践行习近平生态文明思想,坚定贯彻新发展理念,根据党中央、国务院决策部署,按照山水林田湖草系统治理要求\cite{Efthymiou2017Matching},在全国全面推行林长制,明确地方党政领导干部保护发展森林草原资源目标责任,构建党政同责、属地负责、部门协同、源头治理、全域覆盖的长效机制,加快推进生态文明和美丽中国建设。

森林和草原是重要的自然生态系统\cite{luo2018cross},对于维护国家生态安全、推进生态文明建设具有基础性、战略性作用。为全面提升森林和草原等生态系统功能,进一步压实地方各级党委和政府保护发展森林草原资源的主体责任,现就全面推行林长制提出以下意见。

新华社北京1月12日电 近日,中共中央办公厅、国务院办公厅印发了《关于全面推行林长制的意见》,并发出通知,要求各地区各部门结合实际认真贯彻落实\cite{Deng2019Table2Vec}。

\subsection{研究现状总结}

森林和草原是重要的自然生态系统,对于维护国家生态安全、推进生态文明建设具有基础性、战略性作用。为全面提升森林和草原等生态系统功能,进一步压实地方各级党委和政府保护发展森林草原资源的主体责任,现就全面推行林长制提出以下意见。

\begin{enumerate}
    \item 指导思想。以习近平新时代中国特色社会主义思想为指导,全面贯彻党的十九大和十九届二中、三中、四中、五中全会精神,认真践行习近平生态文明思想,坚定贯彻新发展理念,根据党中央
    \item 国务院决策部署,按照山水林田湖草系统治理要求,在全国全面推行林长制,明确地方党政领导干部保护发展森林草原资源目标责任,构建党政同责、属地负责、部门协同、源头治理、全域覆盖的长效机制,加快推进生态文明和美丽中国建设。
\end{enumerate}