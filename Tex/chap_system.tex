\section{系统设计与实现}

\subsection{系统软件模块}

加强森林草原资源生态保护。严格森林草原资源保护管理,严守生态保护红线。严格控制林地、草地转为建设用地,加强重点生态功能区和生态环境敏感脆弱区域的森林草原资源保护,禁止毁林毁草开垦。加强公益林管护,统筹推进天然林保护,全面停止天然林商业性采伐,完善森林生态效益补偿制度。落实草原禁牧休牧和草畜平衡制度,完善草原生态保护补奖政策。强化森林草原督查,严厉打击破坏森林草原资源违法犯罪行为。推进构建以国家公园为主体的自然保护地体系。强化野生动植物及其栖息地保护。

系统的软件模块大致分为三个层级:

\begin{enumerate}
    \item 组织体系。各省(自治区、直辖市)设立总林长,由省级党委或政府主要负责同志担任
    \item 设立副总林长,由省级负责同志担任,实行分区(片)负责。各省(自治区、直辖市)根据实际情况,可设立市、县、乡等各级林长。地方各级林业和草原主管部门承担林长制组织实施的具体工作。\\
    工作职责。各地要综合考虑区域、资源特点和自然生态系统完整性,科学确定林长责任区域。各级林长组织领导责任区域森林草原资源保护发展工作,落实保护发展森林草原资源目标责任制。\\
    将森林覆盖率、森林蓄积量、草原综合植被盖度、沙化土地治理面积等作为重要指标,因地制宜确定目标任务;组织制定森林草原资源保护发展规划计划。
    \item 强化统筹治理,推动制度建设,完善责任机制;组织协调解决责任区域的重点难点问题
\end{enumerate}

\subsection{系统部署}

本系统为坚持生态优先、保护为主。全面落实森林法系统,系统部署的硬件配置如表\ref{Tab:systemhardware}所示:

\begin{inserttable}{系统部署的硬件配置}{Tab:systemhardware}{p{0.15\textwidth}<{\centering}p{0.55\textwidth}<{\centering}}
    \hline
    硬件名称    &   实际配置                            \\
    \hline
    GPU         &   NVIDIA GeForce RTX 2080Ti × 2       \\
    CPU         &	Intel® Core™ i7-8700U CPU @3.2GHz   \\
    内存        &	DDR4 32GB                           \\
    显存        &	12GB                                \\
    硬盘容量    &	1TB                                 \\
    \hline
\end{inserttable}

系统部署的软件配置如表\ref{Tab:systemsoftware}所示:

\begin{inserttable}{系统部署的软件配置}{Tab:systemsoftware}{p{0.2\textwidth}<{\centering}p{0.4\textwidth}<{\centering}}
    \hline
    软件环境    &	实际配置    \\
    \hline
    程序语言    &	Python 3.7  \\
    开发工具    &	VS Code     \\
    并行计算架构&	CUDA 10.2   \\
    机器学习库  &	PyTorch 1.3 \\
    \hline
\end{inserttable}

系统将部署在如图\ref{Fig:realenvironment}所示的实际环境中。

\inserttwograph{hardware.png}{software.png}{系统部署实际环境}{Fig:realenvironment}{硬件机房}{软件配置}

\subsection{系统性能测试}

坚持绿色发展、生态惠民。牢固树立和践行绿水青山就是金山银山理念,积极推进生态产业化和产业生态化,不断满足人民群众对优美生态环境、优良生态产品、优质生态服务的需求:

\subtitle{(1)}{测试数据集:}

坚持问题导向、因地制宜。针对不同区域森林和草原等生态系统保护管理的突出问题,坚持分类施策、科学管理、综合治理,宜林则林、宜草则草、宜荒则荒,全面提升森林草原资源的生态、经济、社会功能。

\subtitle{(2)}{测试方法:}

坚持党委领导、部门联动。加强党委领导,建立健全以党政领导负责制为核心的责任体系,明确各级林(草)长(以下统称林长)的森林草原资源保护发展职责,强化工作措施,统筹各方力量,形成一级抓一级、层层抓落实的工作格局。

\subtitle{(3)}{评测指标:}

\begin{itemize}
    \item 组织体系。各省(自治区、直辖市)设立总林长,由省级党委或政府主要负责同志担任;设立副总林长,由省级负责同志担任。
    \item 实行分区(片)负责。各省(自治区、直辖市)根据实际情况,可设立市、县、乡等各级林长。地方各级林业和草原主管部门承担林长制组织实施的具体工作。
\end{itemize}